\documentclass[12pt,letterpaper]{article}
\usepackage{preamble}
\usepackage{tabularx}
\begin{document}

\section*{Topic: 2.4 Geometric Proofs}
\begin{enumerate}
    \item What do you need to bake a cake? I can think of two major categories. Assume you have tools already. What do you need to write a proof?
    \item Define $\angle A$ and $\angle B$ to be angles, and $m\angle A$ and $m\angle B$ are their respective angles measures. Translate each of the following sentences into equations.
          \begin{itemize}
              \item $\angle A$ and $\angle B$ are vertical angles.
              \item $\angle A$ and $\angle B$ are complimentary angles.
              \item $\angle A$ and $\angle B$ are supplementary angles.
              \item $\angle A$ and $\angle B$ are adjacent angles.
              \item $\angle A$, $\angle B$, and $\angle C$ are the interior angles of a triangle.
          \end{itemize}
    \item Define a line segment $\overline{AB}$ where point $M$ is a midpoint along $\overline{AB}$ write out all the equations possible relating $\overline{AM}$, $\overline{MB}$, and $\overline{AB}$.
    \item Below there is a short proof for showing that $\overline{AB}=2\overline{AM}=2\overline{MB}$ However only the left side is done. Copy the table for your notebook and fill out the reasons.\\\\
          \begin{tabularx}{0.95\textwidth} {
                  | >{\raggedright\arraybackslash}X
                  | >{\centering\arraybackslash}X
                  | >{\raggedleft\arraybackslash}X |}
              \hline
              Statement                                                        & Reason \\
              \hline
              \vspace{0.5mm} 1.) $M$ is a midpoint of $\overline{AB}$          & Given  \\
              \hline
              \vspace{0.5mm} 2.) $\overline{AM}\cong\overline{MB}$             &        \\
              \hline
              \vspace{0.5mm} 3.) $\overline{AM}+\overline{MB}=\overline{AB}$   &        \\
              \hline
              \vspace{0.5mm} 4.) $\overline{AM}+\overline{AM}=\overline{AB}$   &        \\
              \hline
              \vspace{0.5mm} 5.) $2\overline{AM}=\overline{AB}$                &        \\
              \hline
              \vspace{0.5mm} 6.) $\overline{AB}=2\overline{AM}$                &        \\
              \hline
              \vspace{0.5mm} 7.) $\overline{AB}=2\overline{MB}$                &        \\
              \hline
              \vspace{0.5mm} 8.) $\overline{AB}=2\overline{AM}=2\overline{MB}$ &        \\
              \hline
          \end{tabularx}

    \item Draw an angle $\angle AOC$ and draw $\overrightarrow{OB}$ that bisects $\angle AOC$. Write out all the equations possible relating $\angle AOB$, $\angle BOC$, and $\angle AOC$.
    \item Write a two column proof (refer to yesterday's exploration) to solve the following question. $\angle A$ and $\angle B$ are vertical. $m\angle A=6x+4$, and $m\angle B=3x+7$. Determine the value of $m\angle A$ and $m\angle B$.
          %\item Write a two column proof (refer to yesterday's exploration) to solve the following question. $\angle A$ and $\Angle B$ are supplementary. $m\angle A=3x+2$, and $m\angle B=-8x+8$. Determine the value of $m\angle A$ and $m\angle B$.
          %\item Write a two column proof (refer to yesterday's exploration) to solve the following question. $\angle A$ and $\Angle B$ are complimentary. $m\angle A=3x+2$, and $m\angle B=-3x+8$. Determine the value of $m\angle A$ and $m\angle B$. Did something weird happen?
\end{enumerate}
\end{document}