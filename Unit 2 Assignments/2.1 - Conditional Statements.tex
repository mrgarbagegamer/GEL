\documentclass[12pt,letterpaper]{article}
\usepackage{preamble}
\usepackage{tabularx}
\begin{document}

\section*{Topic: 2.1 Conditional Statements}
\begin{enumerate}
    \item A conditional statement is in the form 'If $P$, then $Q$.' This may written as: $$P\to Q$$ Where $P$ is called a hypothesis, and $Q$ is some claim. For example: 'If students do the exploration lab, then class will go by faster'. What is the hypothesis and conclusion?
    \item Write your own conditional statement as a full sentence. Identify the hypothesis and conclusion.
    \item For the purposes of logic, if a statement is sometimes true, it will be considered false since a counterexample exists. A counterexample means you assumed the hypothesis is true, but the counterexample turned out false. i.e. We arrive at the conclusion: $$P\to \neg Q$$ Which means 'If $P$, then not $Q$.' Write an example of a true statement, and false statement. For the false statement provide at least one counterexample.
    \item State whether these statements are true, or false. If false, provide a counterexample.
          \begin{itemize}
              \item 'If today is Monday, then tomorrow is Tuesday.'
              \item 'If you divide a number by itself, then you always get one.'
              \item 'If tomorrow is Monday, then it is currently the weekend.'
          \end{itemize}
    \item The converse of a statement is when the hypothesis and conclusion are swapped. Symbolically the converse reads, 'If $Q$, then $P$' or rather: $$Q\to P$$ Find the converse of the statement you wrote from Q3, or ones from Q5.
    \item The inverse of a statement is when the hypothesis and conclusion are negated. Negation basically means opposite, or not. Symbolically the converse reads, 'If not $P$, then not $Q$' or rather:$$\neg P\to \neg Q$$ Find the converse of the statement you wrote from Q3, or ones from Q5.
    \item The contrapositive of a statement is when the hypothesis and conclusion are swapped and negated. Symbolically the converse reads, 'If not $Q$, then not $P$' or rather: $$\neg Q\to \neg P$$ Find the contrapositive of the statement you wrote from Q3, or ones from Q5.
    \item There are special statements called bi-conditional statements which are true both forwards and backwards. One of the conditional statements in Q5.) is bi-conditional. Which one is it?
\end{enumerate}
\end{document}