\documentclass[12pt,letterpaper]{article}
\usepackage{preamble}
\usepackage{tabularx}
\begin{document}

\section*{Topic: 2.6 Parallel Line Proofs}
\begin{enumerate}
    \item Draw a pair of parallel lines and their transversal. Label the angles in a convenient way. Name the following angle relationships, also describe how those angles are related, whether they are complimentary, supplementary, or congruent. Try not to use your table from yesterday:
          \begin{itemize}
              \item Vertical Angles
              \item Corresponding Angles (1)
              \item Alternate Interior Angles (2)
              \item Alternate Exterior Angles (3)
              \item Same Side Interior Angles (4)
              \item Same Side Exterior Angles (5)
          \end{itemize}
    \item We will assume that Vertical Angles and Linear Pairs are postulates, meaning that we do not have to prove them. The claim is that we can start with any of (1)-(5) and prove another of (1)-(5). Let's prove 'If corresponding angles are congruent, then same side interior angles are supplementary'. Firstly identify a pair of corresponding angles, and a pair of same side interior angles. Hint: Try to pick the angles such that each pair shares one angle, it will shorten the length of the proof. Note that you can't always do this.
    \item It is challenging to give direct instructions for how to write a proof, but when ready feel free to go to one of the poster boards and request for assistance. You may start by drawing a two-column table and writing statements and reasons, and also the first step (given information) if you can.
    \item Pick another of (1)-(5) and prove another of (1)-(5).
    \item Pick one of (2) or (3) and prove the other (2) or (3).
    \item Pick one of (4) or (5) and prove the other (4) or (5).
    \item Pick (1) and prove any of (2)-(5). Do you see a strategy?
    \item Pick one of (2)-(5) and prove (1). Do you see a strategy?
    \item Pick any of (2) or (3) and prove (4) or (5).
    \item Pick any of (4) or (5) and prove (2) or (3).
    \item Which proof did you find the most challenging?
\end{enumerate}
\end{document}