\documentclass[12pt,letterpaper]{article}
\usepackage{preamble}
\usepackage{tabularx}
\begin{document}

\section*{Topic: 1.SG Exploration and Intrigue for Benchmark 1}
\begin{enumerate}
    \item Now that we're done with a unit, we might as well have some fun exploring things. Look back to your past explorations (notes at least, if you don't have the handouts) and feel free to explore something. These exploration questions are not necessarily on the exam, but feature very big ideas of the course in general and will likely inhibit some reminders of related ideas for future units.
    \item Draw a rectangle and measure its perimeter and area. Now draw the midpoints of all the sides and create a new rectangle. Measure its perimeter and area. What do you notice?
    \item Which of these can you not have in 2D space? Parallel lines, congruent lines, perpendicular lines, skew lines. Justify.
    \item Draw a square of side length 2. Alternate shading in the colors of the four squares that make it up, like a chess or checkerboard. How many of each square do you have? Repeat the same for side length 4, 6, and 8. What do you notice? What do you conjecture?
    \item Draw a triangle with generic angles and side lengths. Label this triangle $\Delta$ABC. Rank from smallest to largest the angles and the sides. What do you notice?
    \item Assume you know the area of a parallelogram is $bh$ or base times height. Prove that the area of a triangle is $\dfrac{1}{2}bh$.
    \item Pretend you were asked to solve the following terrifying for 8th grade easy for 10th grade equation: $$x+1(x+2(x+3))=x+4(x+5(x+6))$$ How would you start? How would you tell a past version of yourself on how to solve this? Hint: It's like the opposite of peeling an onion...
    \item Draw a quadrilateral (4 sides) with the property: Four congruent sides. What did you draw? Is there another answer? What is the name of that shape? Is that the most general name for that shape?
    \item Do the same for four congruent angles.
    \item Do the same for two congruent sides.
    \item Do the same for two congruent angles.
    \item Do the same for one pair of parallel sides.
    \item Do the same for two pairs of parallel sides.
\end{enumerate}

\end{document}