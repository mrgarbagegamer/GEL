\documentclass[12pt,letterpaper]{article}
\usepackage{preamble}
\usepackage{tabularx}
\begin{document}

\section*{Topic: 1.6 Segment and Angle Bisectors}
\begin{enumerate}
    \item Name words that have the prefix 'bi-'. What is usually true in these scenarios? What does the prefix 'bi-' mean?
    \item Draw a line segment, label the endpoints. Draw what you think it means to bisect it. Does it match your definition above?
    \item What do you notice about your bisected segment? Label the endpoints and you should also have a midpoint, what do you notice?
    \item Draw an angle, label the endpoints and vertex. Draw what you think it means to bisect it. Does it match your definition above?
    \item What do you notice about your bisected angle?
    \item Draw the straight angle (180 degrees) and bisect it. What angles do you get?
    \item Draw a right angle and bisect it. What angles do you get?
    \item When you are bisecting an angle, what ratio are the segments being split up into?
    \item I am thinking of a line segment of length $10$ What would be the length of the bisected segments?
    \item I am thinking of an angle of $50^\circ$. What would be the angle measure of the bisected angles?
    \item I am thinking of a line segment of length $s$. What would be the angle measure of the bisected segments?
    \item I am thinking of an angle of $\theta$ degrees. What would be the angle measure of the bisected angles?
\end{enumerate}

\end{document}