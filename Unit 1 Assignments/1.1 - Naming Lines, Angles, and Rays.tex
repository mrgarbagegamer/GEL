\documentclass[12pt,letterpaper]{article}
\usepackage{preamble}
\usepackage{tabularx}
\begin{document}

\section*{Topic: 1.1 Naming Lines, Angles, and Rays}
\begin{enumerate}
    \item Draw two points, how can you connect them with the shortest distance possible?
    \item Draw three points, how can you connect them with the shortest distance possible? How did you connect them?
    \item Is there a difference between connecting two points and three points? What about four points?
    \item Draw a line that goes forever in both directions. What trouble might arise?
    \item Draw a ray that only goes forever in one direction.
    \item Draw a line segment. Measure its length by using the width of your index finger (which is roughly a centimeter).
    \item For the line segment, label two points on it. Trace your finger from one point to the other, then the other way. Are these the same line segment?
    \item Draw a line segment containing 4 points. Name the four points after your favorite animal. Name every single possible line segment. You can color code them if you want. How many line segments can you name?
    \item Draw two angles that are the same. Why are they the same? What happens if you rotate or flip it? Are they still the same?
    \item Draw two angles that are different. Why are different? What happens if you rotate or flip it? Are they still the same?
    \item Draw two lines, do they always intersect? What forms when they intersect? Is this always true?
    \item Imagine slicing a paper with another slice of paper. What forms when they intersect? Is this always true?
\end{enumerate}

\end{document}