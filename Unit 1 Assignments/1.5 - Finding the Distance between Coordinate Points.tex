\documentclass[12pt,letterpaper]{article}
\usepackage{preamble}
\usepackage{tabularx}
\begin{document}

\section*{Topic: 1.5 Finding the Distance between Coordinate Points}
\begin{enumerate}
    \item You'll need a calculator and ruler today for sure. Determine the value of these numbers:
          \begin{itemize}
              \item $6^2+8^2$
              \item $5^2+12^2$
              \item $\sqrt{6^2+8^2}$
              \item $\sqrt{5^2+12^2}$
          \end{itemize}
          What do you notice? What do you conjecture? Can you find another 'Pythagorean Triple?' Do you think there are infinitely many? Do all numbers work?
    \item Draw a large right triangle with a ruler. Measure the two small side lengths with a ruler. Square them, then add them, then take a square root of the sum.
    \item Measure the third long side, what did you notice?
    \item Draw another right triangle. Measure the two short sides with a ruler. Predict with the Pythagorean Theorem the length of the third side, then actually measure it.
    \item Draw two 'lattice points' on a graph. A lattice point is a coordinate pair with integer $(\dots -3,-2,-1,0,1,2,3\dots)$ coordinates. Create a right triangle out of them. How can you determine the distance between them? Do this both with Pythagorean Theorem then measure the length with a ruler.
    \item Here is an introduction to a PODASIP (Prove or Disprove, and Salvage if Possible), but we'll do a tiny version.
          Is the following statement true?
          $$\sqrt{a^2+b^2}=a+b$$ Hint: Just try picking numbers for $a$ and $b$ and checking if it's true.
    \item If you're brave, prove whether it's true, and if it's false, find out when the statement is true.
    \item Just to check our algebra chops, solve for $x$ in each.
          \begin{itemize}
              \item $6^2+8^2=x^2$
              \item $9^2+12^2=x^2$
              \item $x^2+20^2=25^2$
              \item $x^2+24^2=30^2$
          \end{itemize}
          What do you notice?
    \item Draw a right triangle. Measure the longest side and another side. Predict with the Pythagorean Theorem the length of the third side, then actually measure it.
\end{enumerate}

\end{document}