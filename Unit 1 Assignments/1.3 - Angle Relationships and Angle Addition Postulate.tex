\documentclass[12pt,letterpaper]{article}
\usepackage{preamble}
\usepackage{tabularx}
\begin{document}

\section*{Topic: 1.3 Angle Relationships and Angle Addition Postulate}
\begin{enumerate}
    \item Last time we discussed the segment addition postulate. Do you think there is a need for a segment subtraction postulate? What would it be?
    \item Make a guess of what an Angle Addition Postulate would be.
    \item Draw an example of congruent lines.
    \item Draw an example of adjacent angles.
    \item Draw an example of vertical angles.
    \item Draw an example of complimentary angles.
    \item Draw an example of supplementary angles.
    \item Draw an acute, right, and obtuse angle.
    \item For the acute angle, add something to it to make it a right angle.
    \item For the right angle, split it in half, what are those angles?
    \item For the obtuse angle, add something to make it a straight line (a.k.a. the straight angle)
    \item Pick your favorite letters from yesterday's exploration, determine what kind of angles are present and use geometric markings to show it.
    \item Draw two non-equal angles that add up to 90. Estimate their angle measures.
    \item Draw three non-equal angles that add up to 90.
          Estimate their angle measures.
    \item Draw two non-equal angles that add up to 180. Estimate their angle measures.
    \item Draw three non-equal angles that add up to 180.
          Estimate their angle measures.
\end{enumerate}

\end{document}