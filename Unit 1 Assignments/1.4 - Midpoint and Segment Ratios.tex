\documentclass[12pt,letterpaper]{article}
\usepackage{preamble}
\usepackage{tabularx}
\begin{document}

\section*{Topic: 1.4 Midpoint and Segment Ratios}
\begin{enumerate}
    \item Draw a line segment and draw a dot where the dot is in the middle of the segment so that it will split the segment into equally halves. This is a midpoint.
    \item Draw a line segment and label the endpoints with the first and last letter of a three letter animal. Draw the midpoint using the middle letter of the three letter animal. Name all the segments possible and describe how they are related.
    \item Draw a line segment and label its endpoints with coordinates. Draw the midpoint and determine the coordinates of the midpoint.
    \item Draw a line segment and split it into three equal sections, is it as easy as splitting the line segment in half? Describe why.
    \item Draw a line segment and split it into four equal sections. Was this easier or harder than splitting it into three? Why?
    \item There are 30 people in a class, they are split into a 2:3 ratio of boys to girls, how many boys and girls are there in the class? How do you know? Can you generalize?
    \item There are 123 people in high school, they are split into a 20:20:1 ratio of lads, gals, and non-binary pals. How many of each are in the high school?
    \item There are 12345678910 people in the world, they are split into a 1:2:3:4 ratio of people liking algebra, geometry, calculus, and physics. Determine how many people like each subject.
    \item Draw a few congruent segments. Split it in a 1:1 ratio, 1:1:1 ratio, 1:1:1:1 ratio, 1:2:1 ratio, 2:1:1 ratio, 1:1:2 ratio, 1:3 ratio, and 3:1 ratio, what do you notice?
\end{enumerate}

\end{document}