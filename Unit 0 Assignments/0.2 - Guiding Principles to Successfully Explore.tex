\documentclass[12pt,letterpaper]{article}
\usepackage{preamble}
\usepackage{tabularx}
\begin{document}

\section*{Topic: 0.2 Guiding Principles to Successfully Explore}
Our guiding principles are \textbf{Exploration}, \textbf{Explanation}, and \textbf{Fun}.

\section*{Exploration} Don’t worry about finishing all the problems, or even finishing a particular problem. If you get partway through a problem, and find your own questions to ask, that’s great! Go for it and answer your own questions. Those are the best kind.

\section*{Explanation} Getting to an answer is only valuable if you are able to communicate your answer to someone else and explain how you got there. Explaining mathematics clearly and precisely is not easy (just ask your math teacher!), but it is important to practice this important skill! Even when something seems “obvious” (especially when something seems obvious, perhaps), make sure you can explain it so others can understand.

\section*{Fun} This one is self-explanatory. Fun, have it!

Here is a brief example of what an exploratory question could be. The next page will have an example exploration.\\
(Note that you would be doing these in your notes, not on this paper)
\begin{enumerate}
    \item Mr. Freeman claims he can read your mind.
          \begin{itemize}
              \item Firstly pick a two digit number.
              \item Then add their digits.
              \item Now subtract your original number from the sum of those digits. \item Now divide that by the digit in the original number's tens place.
              \item He knows what number you're thinking of. Try this for other two digit numbers!
          \end{itemize}
\end{enumerate}

\end{document}