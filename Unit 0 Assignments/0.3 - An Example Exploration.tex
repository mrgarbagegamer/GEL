\documentclass[12pt,letterpaper]{article}
\usepackage{preamble}
\usepackage{tabularx}
\begin{document}

\section*{Topic: 0.3 An Example Exploration}
Here is a brief example of what an exploratory question could be and an example exploration.
(Note that you would be doing these in your notes, not on this paper)
\begin{enumerate}
    \item Mr. Freeman claims he can read your mind.
          \begin{itemize}
              \item Firstly pick a two digit number.
              \item Then add their digits.
              \item Now subtract your original number from the sum of those digits. \item Now divide that by the digit in the original number's tens place.
              \item He knows what number you're thinking of. Try this for other two digit numbers!
          \end{itemize}
\end{enumerate}

\begin{enumerate}
    \item Explore:
          \begin{itemize}
              \item Let's try 47. $4+7=11$, and $47-11=36$. I picked 'forty-seven' so the 4 is in the 10's place. $36\div4=9$.
              \item Let's try 58. $5+8=13$, and $58-13=45$. I picked 'fifty-eight' so the 5 is in the 10's place. $45\div5=9$. Wait I got nine again? I'm going to ask Jonathan and Leila what they found.
          \end{itemize}
    \item Discuss: Jonathan found that using 36 and 27 he also got 9. Leila tried with 71 and 89 and she also got 9.
    \item Conjecture: I claim that whenever you have a two digit number, subtract their sum of digits, then divide by the digit in the 10's place you always get 9. But I don't know how to prove it!
    \item Apply: I wonder if this works for 3 digit numbers. Let's try $123$, $1+2+3=6$, and $123-6=117$. Hmm dividing by 2 doesn't look good as $117\div2=58.5\dots$ I'll try 274. $2+7+4=13$, and $274-13=261$. Hmmm, $\dfrac{261}{7}$ doesn't look like a nice number, but hey I see a pattern between $117$ and $261$ but I won't let Mr. Freeman have all the fun! :)
    \item Extend: There has to be some relationship between having a number and subtracting all of its digits. I wonder why the 2 digit case works so nicely and the 3 digit case doesn't?

\end{enumerate}
\end{document}